\documentclass[12pt, xcolor=table]{beamer}
\usepackage{graphicx}
\usepackage[spanish]{babel}
\usepackage[utf8]{inputenc}
\usepackage{amsmath}
\usepackage{amssymb}
\usepackage{listings}
\usepackage{hyperref}
\usepackage{fancyvrb}
\usepackage{color}
\usepackage{verbatim}
\usepackage{alltt}
\usepackage[round]{natbib}
%\usepackage{babelbib}
\usepackage{csquotes}

\usepackage[percent]{overpic}
\usepackage[footnotesize, bf]{caption}
%\input{theme.tex}
\usetheme{CambridgeUS}
\usecolortheme{seagull}
%\input{syntax}
\renewcommand{\footnotesize}{\tiny}

\begin{document}
\title{Redes comunitarias wifi y el proyecto Freifunk}
\author{Lusy}
\date{\today}
%add more Info: examen oral, VM, ...

\begin{frame}
    \titlepage
\end{frame}

\begin{frame}
    \frametitle{Visión general}
    \tableofcontents
\end{frame}

\section{Definición}
\begin{frame}
  \frametitle{Definición: redes comunitarias wifi}
    \begin{itemize}
      \item el principio de la neutralidad de la red
      \item infraestructura abierta, accesible para tod@s
      \item ejemplos: Freifunk, guifi.net, ninux, Funkfeuer
      \item add logos?
    \end{itemize}
\end{frame}

\section{La parte técnica}
\begin{frame}
  \frametitle{La parte técnica: ¿cómo funciona?}
    \begin{itemize}
      \item nodos de la red: routers
      \item software especial
      \item software libre desarollado por la comunidad (qué es software libre)
      \item una red mesh vs la topografía estandard (estrella)
      \item add picture of mesh network
      \item cada nodo está conocido dentro de la red
      \item cada nodo sabe como puede enviar datos hasta el nodo destinatario
    \end{itemize}
\end{frame}

\section{Motivación}
\begin{frame}
  \frametitle{Motivación: ¿por qué participa la gente?}
  \begin{itemize}
    \item profundizar los propios conocimientos técnicos
    \item experimentar, Bildungsauftrag,..
  \end{itemize}
\end{frame}

\section{Freifunk}
\begin{frame}
  \frametitle{Un ejemplo: el proyecto Freifunk}
  \begin{itemize}
    \item comunidades locales por toda Alemania (picture)
    \item la estructura en Berlín: backbones, etc
    \item principios de la comunidad Freifunk?
    \item conectar a residencias de asilad@s/refugiad@s
    \item add logo
  \end{itemize}
\end{frame}

\begin{frame}
  \frametitle{Freifunk en Berlín}
  mapa
\end{frame}

\begin{frame}
  \frametitle{¿Cómo se puede participar?}
  \begin{itemize}
    \item instalar el sistema operativo (firmware) libre en un router y ponerlo en casa
    \item desarrollar el firmware
    \item hacer instalaciones técnicas en los techos de Berlín
    \item y mantener la red
    \item difundir la idea, hacer publicidad
    \item enseñar a otr@s como funciona el proyecto
  \end{itemize}
\end{frame}

\begin{frame}
  \frametitle{¿Para qué se puede usar la red?}
  \begin{itemize}
    \item obtener acceso a Internet (sobre todo en casos, en los que está difícil con los proveedores de servicios Internet convencionales)
    \item prestar y usar servicios locales
      \begin{itemize}
        \item telefonear dentro de la red Freifunk
        \item cámara para la observación de pájaros
      \end{itemize}
  \end{itemize}
\end{frame}

\section{Fuentes}
\begin{frame}
  \frametitle{Fuentes}
  \nocite{*}
  \bibliography{literatura}
  \bibliographystyle{unsrtnat}
\end{frame}

\end{document}
