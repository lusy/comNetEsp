\documentclass[12pt, xcolor=table]{beamer}
\usepackage{graphicx}
\usepackage[spanish]{babel}
\usepackage[utf8]{inputenc}
\usepackage{amsmath}
\usepackage{amssymb}
\usepackage{listings}
\usepackage{hyperref}
\usepackage{fancyvrb}
\usepackage{color}
\usepackage{verbatim}
\usepackage{alltt}
\usepackage[round]{natbib}
%\usepackage{babelbib}
\usepackage{csquotes}

\usepackage[percent]{overpic}
\usepackage[footnotesize, bf]{caption}
%\input{theme.tex}
\usetheme{CambridgeUS}
\usecolortheme{seagull}
%\input{syntax}
\renewcommand{\footnotesize}{\tiny}

\begin{document}
\title{Redes comunitarias wifi y el proyecto Freifunk}
\author{Lusy}
\date{\today}

\begin{frame}
    \titlepage
\end{frame}

\begin{frame}
    \frametitle{Visión general}
    \tableofcontents
\end{frame}

\section{Definición}
\begin{frame}
  \frametitle{Definición: redes comunitarias wifi}
    \begin{itemize}
      \item ejemplos: Freifunk, guifi.net, ninux, Funkfeuer
    \end{itemize}
\end{frame}

\section{Motivación}
\begin{frame}
  \frametitle{Motivación: ¿por qué participa la gente?}
  \begin{itemize}
    \item el principio de la neutralidad de la red
    \item infraestructura abierta, accesible para tod@s
    \item profundizar los propios conocimientos técnicos
    \item experimentar, Bildungsauftrag,..
  \end{itemize}
\end{frame}

\section{La parte técnica}
\begin{frame}
  \frametitle{La parte técnica: ¿cómo funciona?}
    \begin{itemize}
      \item una red mesh vs la topografía estandard (estrella)
    \end{itemize}
\end{frame}

\section{Freifunk}
\begin{frame}
  \frametitle{Un ejemplo: el proyecto Freifunk}
  \begin{itemize}
    \item comunidades locales por toda Alemania
    \item la estructura en Berlín: backbones, etc
    \item conectar a residencias de asilad@s/refugiad@s
  \end{itemize}
\end{frame}

\begin{frame}
  \frametitle{Freifunk en Berlín}
  mapa
\end{frame}

\begin{frame}
  \frametitle{¿Cómo se puede participar?}
\end{frame}

\section{Fuentes}
\begin{frame}
  \frametitle{Fuentes}
\end{frame}

\end{document}
